\documentclass[]{article}
\usepackage{lmodern}
\usepackage{amssymb,amsmath}
\usepackage{ifxetex,ifluatex}
\usepackage{fixltx2e} % provides \textsubscript
\ifnum 0\ifxetex 1\fi\ifluatex 1\fi=0 % if pdftex
  \usepackage[T1]{fontenc}
  \usepackage[utf8]{inputenc}
\else % if luatex or xelatex
  \ifxetex
    \usepackage{mathspec}
  \else
    \usepackage{fontspec}
  \fi
  \defaultfontfeatures{Ligatures=TeX,Scale=MatchLowercase}
\fi
% use upquote if available, for straight quotes in verbatim environments
\IfFileExists{upquote.sty}{\usepackage{upquote}}{}
% use microtype if available
\IfFileExists{microtype.sty}{%
\usepackage{microtype}
\UseMicrotypeSet[protrusion]{basicmath} % disable protrusion for tt fonts
}{}
\usepackage[margin=1in]{geometry}
\usepackage{hyperref}
\hypersetup{unicode=true,
            pdftitle={Simulation effects of climate change on fire regime: implications for Boreal Caribou and landbird communities in the Northwest Territories},
            pdfborder={0 0 0},
            breaklinks=true}
\urlstyle{same}  % don't use monospace font for urls
\usepackage{longtable,booktabs}
\usepackage{graphicx,grffile}
\makeatletter
\def\maxwidth{\ifdim\Gin@nat@width>\linewidth\linewidth\else\Gin@nat@width\fi}
\def\maxheight{\ifdim\Gin@nat@height>\textheight\textheight\else\Gin@nat@height\fi}
\makeatother
% Scale images if necessary, so that they will not overflow the page
% margins by default, and it is still possible to overwrite the defaults
% using explicit options in \includegraphics[width, height, ...]{}
\setkeys{Gin}{width=\maxwidth,height=\maxheight,keepaspectratio}
\IfFileExists{parskip.sty}{%
\usepackage{parskip}
}{% else
\setlength{\parindent}{0pt}
\setlength{\parskip}{6pt plus 2pt minus 1pt}
}
\setlength{\emergencystretch}{3em}  % prevent overfull lines
\providecommand{\tightlist}{%
  \setlength{\itemsep}{0pt}\setlength{\parskip}{0pt}}
\setcounter{secnumdepth}{0}
% Redefines (sub)paragraphs to behave more like sections
\ifx\paragraph\undefined\else
\let\oldparagraph\paragraph
\renewcommand{\paragraph}[1]{\oldparagraph{#1}\mbox{}}
\fi
\ifx\subparagraph\undefined\else
\let\oldsubparagraph\subparagraph
\renewcommand{\subparagraph}[1]{\oldsubparagraph{#1}\mbox{}}
\fi

%%% Use protect on footnotes to avoid problems with footnotes in titles
\let\rmarkdownfootnote\footnote%
\def\footnote{\protect\rmarkdownfootnote}

%%% Change title format to be more compact
\usepackage{titling}

% Create subtitle command for use in maketitle
\providecommand{\subtitle}[1]{
  \posttitle{
    \begin{center}\large#1\end{center}
    }
}

\setlength{\droptitle}{-2em}

  \title{Simulation effects of climate change on fire regime: implications for
Boreal Caribou and landbird communities in the Northwest Territories}
    \pretitle{\vspace{\droptitle}\centering\huge}
  \posttitle{\par}
    \author{}
    \preauthor{}\postauthor{}
      \predate{\centering\large\emph}
  \postdate{\par}
    \date{Draft last revised 3/31/2019}

\usepackage{color}

\begin{document}
\maketitle

{
\setcounter{tocdepth}{3}
\tableofcontents
}
\section{Executive Summary}\label{executive-summary}

The purpose of this project was to support multi-species modelling
objectives of Environment and Climate Change Canada with respect to
woodland caribou and migratory songbirds. We sought to integrate models
of vegetation dynamics, wildfire, and anthropogenic change with
statistical models of songbird abundances and woodland boreal caribou
(\emph{Rangifer tarandus caribou}) resource selection and demographic
parameters. The integration framework is SpaDES (Chubaty and McIntire
2017), a system of R packages for statistical and geospatial analysis
and spatial simulation, designed for ecological forecasting applications
over very large areas. The integrated model is used to forecast the
spatial distribution of selected songbird species and the regional
potential to sustain viable populations of woodland boreal caribou,
under climate change over the 21st century. Other potential applications
include quantifying the overlap between areas of high conservation value
for birds and for caribou, and evaluating the conservation potential in
these respects of specific candidate protected areas. The spatial extent
of this work is Bird Conservation Region (BCR) 6 as contained within the
Northwest Territories of Canada. The integrated model is implemented as
a suite of eight groups of SpaDES modules. A common spatial resolution
of 250m is used in all modules. Most reporting is done over forested
habitats only, neglecting waterbodies, wetlands, and areas of tundra
vegetation or shrubands (see the \texttt{LandR} module description, and
future directions, below).

This project is, to our knowledge, the most comprehensive and extensive
integration of landscape and climate change effects on Species At Risk
to date. It represents a novel and important contribution to ecological
forecasting, and contains the ability to uniquely address both national
(i.e. \emph{Species At Risk Act}) and international (i.e.
\emph{Convention on Biological Diversity Aichi Target 11}) policy
mandates for biodiversity conservation today, and under anticipated
future conditions.

The models contained within this project represent regional processes
and are intended to support regional planning initiatives. The models
incorporpate the best current understanding of ecological processes, and
make use of statistical relationships at regional and national scales.
They may not acurately predict localized processes. Accordingly, the
results should not be taken as predictions of the specific conditions at
any given location and time. This discrepency between regional and local
scales is well documented in ecology (Wiens 1989; Holling 1992; Levin
1992; Burton et al. in review) with direct ramifications on downscaling
models (e.g.~Riiters 2005; Araujo et al. 2005).

The work was conducted under EC Contract No. 30000675933. The final
deliverable under this contract is intended to be a manuscript for
publication in the primary literature. This report represents a draft of
the Methods Section of this target manuscript, with some preliminary
results.

\subsection{General project components as stated in the
RFP}\label{general-project-components-as-stated-in-the-rfp}

Item \#1: Species Abundance Models (SAMs) for landbirds under current
conditions. Item \#2: Land change simulation (2000-2100). Item \#3:
Demographic models for caribou, SAMs for landbirds, co-occurrence of
caribou and landbirds under future conditions.

\section{Integrated modules overview: Items 1, 2, and components of item
3}\label{integrated-modules-overview-items-1-2-and-components-of-item-3}

This integrated model is implemented as a SpaDES metamodule. The
metamodule is composed of eight named groups of tightly coupled modules;
there are are known as ``parent modules'' in SpaDES parlance. Each
parent implements a key model process (e.g.~fire, vegetation dynamics)
or component (e.g.~bird abundance predictions). For an overview of the
concept of SpaDES modules, see
\url{https://cran.rproject.org/web/packages/SpaDES.core/vignettes/ii-modules.html}.
The modules are structured as follows: {[}Figure 1. Module structure
that composes the NWT project{]}
(\url{https://github.com/tatimicheletti/NWT/raw/master/inputs/ModelStructure_2.jpg})

Running the full metadmodule for a 100 year simulation over the entire
study area can take many days on a 1 TB, 50 core RAM Linux computer.

\begin{longtable}[]{@{}cc@{}}
\toprule
module & approximate duration\tabularnewline
\midrule
\endhead
LandR & 3.05 days\tabularnewline
scfm-fireSense &\tabularnewline
anthropogenic &\tabularnewline
BirdsNWT & 6.25 days\tabularnewline
caribouRSFModel & 1.40 days\tabularnewline
caribouPopGrowthModel & 0.04 days\tabularnewline
comm\_metricsNWT &\tabularnewline
Edehzhie &\tabularnewline
TOTAL DURATION &\tabularnewline
\bottomrule
\end{longtable}

The structure and methods used for each one of the modules is as
follows:

\textcolor{blue}{LandR}: \texttt{LandR} is a spatially explicit model of
vegetation dynamics. It models the biomass of cohorts of tree species
within cells, which interact over a landscape by the process of seed
dispersal. The \texttt{LandR} group of modules is composed of one data
treatment and parameterisation module, and a group of modules that
simulate biomass succession dynamics. The paramaters goverening biomass
change are estimated for the study area specified. The methods and
results of these estimations can be found in Appendix 1.

The first \texttt{LandR} module, \texttt{Boreal\_LBMRDataPrep},
calculates site-specific parameters needed by the biomass succession
modules - for this project these site specific parameters are estimated
for Bird Conservation Region (BCR) 6 within the Northwest Territories.
For instance, maximum biomass, above-ground net primary productiviey
(aNPP), and establishment probabilities from seed germination are
calculated separately depending on tree species cover and biomass,
ecodistrict, land cover class (from the Land Cover Map of Canada 2005;
LCC2005), and stand age. This module also provides other parameters,
such as species tolerances to shade, and other plant traits
(e.g.~longevity, ability to resprout, etc.). These traits were obtained
from trait tables used in LANDIS-II, a popular forest-landscape
simulation model, and are available from Dominic Cyr's GitHub page:
(\url{https://raw.githubusercontent.com/dcyr/LANDIS-II_IA_generalUseFiles/master/speciesTraits.csv})

The group of modules, named \texttt{LandR\_Biomass} form the core of the
vegetation succession model. These modules began as a re implementation
in SpaDES of the LANDIS-II Biomass Succession module (v3.2.1), but have
since been changed in several ways (Barros et al. in prep). They consist
of (1) \texttt{LBMR}, the \texttt{LandR} module responsible for
vegetation aging, dispersal, updating biomass folowing other modules'
events, and producing summary figures and tables; (2)
\texttt{LandR\_BiomassGMOrig}, the \texttt{LandR} module responsible for
cohort growth and mortality; and (3) \texttt{Biomass\_regeneration}, the
\texttt{LandR} module which handles post-disturbance biomass
regeneration (e.g.~fire; Appendix 1). \texttt{LandR} needs at least the
first two modules (\texttt{LBMR} and \texttt{LandR\_BiomassGMOrig}) to
produce sensible vegetation dynamics.

In brief, the \texttt{LandR} modules simulate biomass changes by cohort
(species-age combinations) as a function of age, between-cohort
competition for light, seed dispersal and germination, regeneration
following a disturbance, and mortality due to senescence, competition or
distrubance.

\emph{Current limitations of the \texttt{LandR} module} The purpose of
the module \texttt{Boreal\_LBMRDataPrep} is to estimate parameters of
the vegetation succession modules directly from data. This is done
``automatically'' should the data or study area change. At the moment,
none of the modules within the \texttt{LandR} group automatically
calibrate themselves. We have identified some limitations arising from
this:

\begin{enumerate}
\def\labelenumi{\arabic{enumi}.}
\item
  \emph{Initial decrease in biomass in forested pixels}. Initial
  decreases in biomass during the first decades of the simulation are
  most likely due to starting ages
  (\includegraphics{http://tree.pfc.forestry.ca/kNN-StructureStandVolume.tar})
  being too close to species' longevity parameters (from LANDIS-II
  Biomass Succession v6.2 traits table). This results in many cohorts
  either dying immediately, or almost immediately, at the beginning of
  the simulation leading to a decrease in total biomass. Stand age maps
  and tree longevity are known to be flawed. On the one hand, estimating
  stand age can be tricky and often relies on allometric relationships
  which can vary immensely across environmental gradients. On the other
  hand, tree longevity is frequently estimated from the average maximum
  ``observed'' ages (i.e.~realized age) for a given species; observed
  age is greatly dependent on the environment and disturbance conditions
  to which populations are exposed, rather than the potential maximum
  age of a tree under favourable conditions. There isnt a clear answer
  as to what is the best way of dealing with these issues, but see below
  for some potential suggestions.
\item
  Because the \texttt{LandR} model is desiged to mode the dynamics tree
  species on pathes of forest. It does not represnet shrubs, grasses or
  non-vascular plants, and can not reproduce the processes that cause
  large areas of the northern forest to be dominated by these taxa.
  Therefore, pixels initially assigned to non-forested land-cover
  classes are assumed to remain in that state, indpendent of
  \texttt{LandR} predictions. Model results are summarised only for
  pixels that were in forested classes initially.
\end{enumerate}

Future versions of the \texttt{LandR} module group could include: 1. A
spin-up similar to LANDIS-II (simulating cohort growth from time 0,
minus stand age) to address the potential maximum tree age and biomass
conditions. This also comes with its own set of problems, as only
even-aged stands (single cohorts) can be generated in this way and
biomasses will still need to be adjusted at the end of the spin-up so
that they match more closely the observed biomass. This can be
especially tricky in cases where the current biomass was measured in
forests with complex age structure and species admixture. An alternate
solution may be to let the model run for a few hundred timesteps
(\emph{i.e.} a stabilization phase) until vegetation dynamics reach
quasi-equilibrium, before introducing the subsequent modules within this
project that rely on results from \texttt{LandR} (\emph{i.e.}
\texttt{birdsNWT}, \texttt{caribouRSFModel}, and
\texttt{caribouPopGrowthModel}). These solutions would address current
\texttt{LandR} limitation \#1.

\begin{enumerate}
\def\labelenumi{\arabic{enumi}.}
\setcounter{enumi}{1}
\tightlist
\item
  Models such as \texttt{LandR} will often overestimate tree cover and
  biomass in open habitats because they lack quantified ecological
  mechanisms that limit tree growth in these areas (\emph{i.e.}
  differences in growing season length, soil nutrient and water
  availability, or herbivory). Future versions of the \texttt{LandR}
  modules could (1) ignore this limitation and restrict simulation
  dynamics to currently forested pixels; (2) incorporate estimates for
  these mechanisms allowing \texttt{LandR} to predict for areas that do
  not correspond to purely forested areas; or (3) set a maximum biomass
  limit for \texttt{LandR} projections. These changes would address
  current \texttt{LandR} limitation \#2.
\end{enumerate}

\textcolor{blue}{scfm-fireSense}: \texttt{scfm-fireSense} is a hybrid of
two structurally similar landscape fire models, \texttt{scfm} and
\texttt{fireSense}. In each model, wildfire is simulated as a process of
fire ignition (or ``arrival'', the occurrence of a detected, recorded
fire) and fire spread on a raster grid, in this case the 250m resolution
extent determined by the model template \texttt{rasterToMatch}. The
overall approach is a variant of the class of landscape fire models
based on a simulated percolation process (Hargrove et al. 2000). In
percolation models, fire spread iteratively from burning cells to
unburned neighbours with some probability, in the simplest case this
probability is constant in time and space. Fires are extinguished when
no further spreading occurs. The arrival (or ignition) process is
modelled by stating fires within cells according to some probability.

The original version of \texttt{scfm} is described by Cumming et al.
(1998), with a more accessible version in Armstrong and Cumming (2003).
It was recently implemented as a collection of SpaDES modules by
Cumming, McIntire, and Eddy (in prep), with the addition of automated
parameter estimation for fire management records. Scfm models fire as a
three stage stochastic process of ignition, escape and spread, with each
process represented by a dedicated module. Ignition and spread have
already been described. The escape stage models the effect of fire
suppression, or other ecological or sampling effects that alter the
lower end of the fire size distribution. The empirical quantity is the
``escape probability'', the probability that a fire will attain a final
size greater than the size of a single pixel. This is simulated in the
model by determining an initial spread probability such that the
probability that a fire stays within its cell of origin equals the
escape probability, after accounting for the effect of lakes and other
non-flammable geographic features on fire spread. In effect, this
distinguishes the first step of the iterative fire spread process. These
three modules are used to simulate a fire regime, in terms of the number
of fires that start, the escape probability, and the mean fire size.

In addition to the Ignition, Escape and Spread modules, scfm makes use
of several support modules involved in parameter estimation and model
calibration. Fire regime parameters for ignition, escape, and a mean
fire size, are estimated from historic fire data obtained from the
Canadian National Fire Database (Canadian Forest Service, n.d.). Spread
probabilities are tuned to replicate the empirical mean fire size by a
newly implemented calibration procedure. First, the study region raster
is buffered by a set distance and a flammability map is generated for
the buffered region using the 2005 Land Cover Map of Canada (Latifovic
et al., 2005). Landcover classes such as open water, rock, and ice are
classed as non-flammable. Next, several thousand fires are ignited at
random locations in landscape and spread with probabilities randomly
sampled from a given range (typically 0.18 to 0.24) of spread
probabilities. Fires do not start within the buffer area, but may spread
to and from the buffer. This effectively removes the influence of edge
effects in determining mean fire size, provided the buffer is ``wide
enough''. Then the spread probabilities and resulting fire sizes are fit
with a shape-constrained additive model (SCAM) (Pya and Wood, 2015). The
SCAM is monotonic to ensure fire size increases for any incremental
increase in spread probability. Lastly, a function minimiser, coupled
with scam predict method, is used to find the spread probability that
will reproduce the estimated mean fire size for the region.
Parameterization and calibration of the model can be done separately for
all the polygons in a given shapefile (e.g., by ecoregion), allowing
spatial variation in fire regime parameters among regions.

This version of the model considers lightning-caused fires only, which
is not a serious limitation within BCR6, based on analysis of historical
fire records from 1965-2016. The spread model returns an object to the
sim environment that identifies all burned cells. In many applications,
this would be inputs to other modules that maintain a forest age
structure or updated vegetation state to simulate the effects of fire.
This model does not simulate variation in fire severity, so typically
fires are presumed to be stand initiating. These processes result in
irregular patches of variously sized burns. The model tracks the number
and size of simulated fires in each time step. The fire spread module is
such that considerable interannual variation in area burned is observed
even when the fire model parameters are constant. However, \texttt{scfm}
does not account for effects of vegetation type (other than nonflammable
types) or of fire weather on fire model parameters.

The \texttt{scfm-fireSense} model can be regarded as a generalisation of
\texttt{scfm} where fire ignition and escape probability vary according
to vegetation / land-cover type and fire weather. This is implemented by
generating parameter maps based on the current year's vegetation and
fire weather, so that parameters vary in space and time.

\texttt{scfm-fireSense} is a landscape fire simulation module where fire
regime attributes are sensitive to both climate and vegetation; it
reproduces the spatial and temporal variation of both the number of
fires, and fire escape probability. \texttt{fireSense} is composed of
two groups of modules; a group that prepares and formats the data
(\texttt{climate\_NWT\_DataPrep}, \texttt{fireSense\_NWT\_DataPrep}, and
\texttt{MDC\_NWT\_DataPrep}), a module that translates between
\texttt{LandR} vegetation state and \texttt{fireSense} fueltypes
(\texttt{LBMR2LCC\_DataPrep}), and a group of modules that implements
the algorithms needed to simulate fire
(\texttt{fireSense\_FrequencyFit}, \texttt{fireSense\_FrequencyPredict},
\texttt{fireSense\_EscapeFit}, and \texttt{fireSense\_EscapePredict}).

Briefly, the \texttt{scfm-fireSense} data preparation modules consist
of:

\texttt{climate\_NWT\_DataPrep} downloads climate data from the
AdaptWest project website, performs GIS operations on climate layers and
makes them available to other modules.

\texttt{MDC\_NWT\_DataPrep} computes the Monthly Drought Code from
annual climate data following equations in Bergeron et al. (2010) and
makes the resulting layer available to other modules.

\texttt{fireSense\_NWT\_DataPrep} performs GIS operations on fire and
vegetation data, as well as a number of operations to prepare the data
for analysis/prediction and makes the resulting datasets available to
other modules.

\texttt{LBMR2LCC\_DataPrep} translates \texttt{LandR\_Biomass} outputs
into seven of the nine land cover classes as needed by
\texttt{fireSense} (Table 1); the other two classes are the ``non-fuel''
class, which is assumed to be static in time , and the ``recently
disturbed'' class which is not based on biomass but on age. These fuel
types are defined by an iterative partitioning procedure starting from a
single fuel type containing all vegetation classes, continually refined
until no additional fuel types could be statistically distinguished
based the fire ignition model. fireSense depends on a fueltype a
classifier built within the XGBoost library (Chen and Guestrin 2016)
that predicts fuel type from LandR vegetation state. Table 2 summarises
the accuracy per land cover class for the validation dataset. The
non-fuel class is equivalent to the distinction between flammable and
nonflammable cells used in scfm, as described above. Pixels are
considered disturbed if the time since last disturbance is less than 16
years.

{[}Table 1. Correspondence between the land cover classes used by
\texttt{fireSense} and code from the Land Cover Map of Canada 2005{]}
\textbar{}Land cover class \textbar{} Land cover map of Canada 2005
original code(s) \textbar{}
\textbar{}:-------------------------:\textbar{}:-----------------------------------------------------------\textbar{}
\textbar{}Conifer medium-density \textbar{} 7 \textbar{} \textbar{}Herbs
and shrubs \textbar{} 16, 17, 18, 21, 22, 23, 24, 25, 26, 27, 28, 29,
30, 31, 32 \textbar{} \textbar{}Mixedwood conifer-dominated\textbar{} 13
\textbar{} \textbar{}Non-fuel \textbar{} 33, 36, 37, 38, 39 \textbar{}
\textbar{}Open conifer \textbar{} 20 \textbar{} \textbar{}Other conifer
\textbar{} 1, 6, 8, 9, 10 \textbar{} \textbar{}Other treed \textbar{} 2,
3, 4, 5, 11, 12, 14, 15 \textbar{} \textbar{}Recently disturbed
\textbar{} 34, 35 \textbar{} \textbar{}Wetlands \textbar{} 19 \textbar{}

{[}Table 2. Classifier accuracy per landcover class on the validation
dataset{]} \textbar{}Land cover class \textbar{} Accuracy\textbar{}
\textbar{}:-------------------------:\textbar{}:--------\textbar{}
\textbar{}Conifer medium-density \textbar{} 0.68 \textbar{}
\textbar{}Herbs and shrubs \textbar{} 0.79 \textbar{}
\textbar{}Mixedwood conifer-dominated\textbar{} 0.72 \textbar{}
\textbar{}Open conifer \textbar{} 0.70 \textbar{} \textbar{}Other
conifer \textbar{} 0.57 \textbar{} \textbar{}Other treed \textbar{} 0.79
\textbar{} \textbar{}Wetlands \textbar{} 0.82 \textbar{}

Briefly, the \texttt{scfm-fireSense} simulation modules consist of:

\texttt{fireSense\_FrequencyFit} is an implementation of the method
described in Marchal et al. (2017); it fits a statistical model used to
parameterize the fire ignition component of \texttt{fireSense}. Marchal
et al. (2017) introduced climate sensitivity using the Monthly Drought
Code (MDC) of July and introduced vegetation sensitivity using five
vegetation classes. In this project, we use the MDC of June along with
nine vegetation classes (fuel types) of Table 1, all of which are
derived from the 39 classes of LCC2005 (Table 1). We did not estimate
the fire frequency for the ``non-fuel'' class as it will not burn - it
is defined to be 0. This module outputs a fitted model object of class
\texttt{fireSense\_FrequencyFit}.

\texttt{fireSense\_FrequencyPredict} uses the model object provided by
\texttt{fireSense\_FrequencyFit} to predict fire frequency, or rates of
fire counts. The output of this module is a probability surface
describing the expected fire frequency at a 250 m resolution. This
probability surface is updated annually, and used to feed the ignition
component of \texttt{fireSense}.

\texttt{fireSense\_EscapeFit} fits a statistical model used to
parameterize the fire escape component of \texttt{fireSense}; a fire
escapes when it propagates outside the pixel in which it started. We
used logistic regression to introduce climate and vegetation sensitivity
to the fire escape component. We expressed the probability that a fire
escapes as a function of the Monthly Drought Code of June, and the nine
classes of land cover (Table 1). This module outputs a fitted glm model
object.

\texttt{fireSense\_EscapePredict} predicts fire frequency, or rates of
fire counts, using the model object provided by
\texttt{fireSense\_EscapeFit}. The output of this module is a
probability surface describing the expected probability of fire escape
at a 250 m resolution. This probability surface is updated annually and
used to feed the escape component of \texttt{fireSense}.

\texttt{scfm-fireSense} uses the probability surface predicted by
\texttt{fireSense\_FrequencyPredict} to determine where to start fires.
At the spatial and temporal scale used in this study, the probability of
two or more fires per year is negligible, thus one may use the Bernoulli
approximation, as in Armstrong and Cumming (2003). This parameter map is
used to start fires by sampling uniform random variables for each
flammable pixel, and starting a fire when the random value is smaller
than the Bernoulli probability. For each ignition, fire escapes are
evaluated using the probabilities estimated by
\texttt{fireSense\_EscapePredict}. The state of all escaped fires is
passed to \texttt{scfmSpread} (see below), which spreads the fires using
the per-polygon spread probabilities.

At the spatial and temporal scale used in this study, the probability of
two or more fires per year is negligible, thus one may use the Bernoulli
approximation, as in Armstrong and Cumming (2003). From those fires that
started, \texttt{fireSense} determines which fires escape.
\texttt{scfm-fireSense} then passes the information on pixels that are
burning to \texttt{scfmSpread} (see below), which spreads the fires
using the per-polygon spread probabilities..

\emph{Assumptions of the \texttt{scfm-fireSense} module} 1. The number
of fires per unit area has a negative binomial distribution conditional
on land-cover class and fire weather.

\begin{enumerate}
\def\labelenumi{\arabic{enumi}.}
\setcounter{enumi}{1}
\item
  The fire detection process was constant over the period of record.
\item
  The statistical relationship between fire regime parameters, fire
  weather, and vegetation, will remain constant over the simulation
  period.
\item
  The land cover class ``non-fuels'' is static throughout the simulation
  period, because it represents permanent and nonflammable landscape
  features such as rock outcrops, urban areas, and water.
\end{enumerate}

\emph{Limitations of scfm-fireSense} The full version of the
\texttt{fireSense} module has spatially and temporally variable spread
parameters that vary with fueltype and annual fire weather. It was not
possible to test the software needed to estimate these parameters within
the scope of this contract. Therefore, \texttt{scfm-fireSense} is less
sensitive to variation in vegetation and to changes in fire weather than
the completet fireSense model would be. Experience shows, however, that
the vegetation and climate sentitivity of the ignition and escape
modules produces allready considerable spatial and temporal variation in
patterns of area burned.

\textcolor{blue}{anthropogenic}: The \texttt{anthropogenic} parent
module aims to represent current and project future anthropogenic
disturbances, and to produce metrics of anthropegenic disturbance as
required by other modiles. These metrics include:

\begin{enumerate}
\def\labelenumi{\arabic{enumi}.}
\tightlist
\item
  road density in km/km2 per pixel caclulated over a radius of 10km
\item
  maintain a disturbance layer defined by the union of 500m buffers
  around all anthropogenic disturbances
\end{enumerate}

There are three submodules: \texttt{anthrDisturbanceNWT},
\texttt{lineDensity}, and \texttt{Buffer}. It provides information to
the \texttt{birdsNWT}, \texttt{caribouPopGrowth}, and
\texttt{caribouRSFModel} modules.

The \texttt{anthrDisturbanceNWT} module loads, prepares, and merges all
anthropogenic disturbance vector data layers into a layer called
anthrDisturb. This layer is updated by adding predefined disturbance
features as indicated in a feature update schedule file
anthrDisturbSchedule.txt, found in the module's data directory. The file
specifiies a planning schedule or scenario for the addition of new
featiures. It contains columns with information on planned timing,
feature selection, source url and other feature descriptives
(i.e.~planTime, Class, name, shortName, IDcol, featureID, targetFile and
url) of the future disturbances. Currently, only the proposed extention
of the McKenzie highway is included.

The \texttt{anthrDisturbanceNWT} module initializes
\texttt{antheDisturb} from two data sources:

\begin{enumerate}
\def\labelenumi{\arabic{enumi}.}
\tightlist
\item
  the ``Boreal ecosystem anthropogenic disturbance'' (BEADS) vector data
  for anthropogenic features. BEADS is current to 2008-2010. It is is
  available from
  (\url{https://open.canada.ca/data/en/dataset/afd0ce47-17c3-445c-b823-2f86409da2e0});
\item
  The 2010 road network file for the NWT (NWT ROADS) is available from
  (\url{https://open.canada.ca/data/en/dataset/ab807d3f-9112-4d67-b17b-12bf83ff81e2}).
\end{enumerate}

There are seven feature classes:

\begin{enumerate}
\def\labelenumi{\arabic{enumi}.}
\tightlist
\item
  airstrips (BEADS)
\item
  pipelines (BEADS)
\item
  powerlines (BEADS)
\item
  railways (BEADS)
\item
  roads (NWT ROADS)
\item
  seismic lines (BEADS)
\item
  unknown (BEADS)
\end{enumerate}

The \texttt{lineDensity} module computes the total density all specified
linear feature classes within a specified radius around each raster
cell, expressed as magnitude per-unit area (km/km2). The default input
object is anthrDisturb, from which features can be selected. In this
model only the roads layer of anthrDistur is used. The default radius is
10 km, as per the requirements of the caribou RSF model. The road
density output LFDensityMap is masked to waterRaster, such that water
cells have no road density.

The direct method of calculating \texttt{lineDensity} would be extrenely
compute-intensive, requiring to define a circular buffer of 40 pixel
radius around each pixel. We use an approximation. First, lengths of
linear features within each raster cell are calculated. The values for
line length within each raster cell are then spread out over all
neighbouring cells within a 10km radius and summed where they overlap
using fast fourier transforms (fft). This is equivalent to summing total
line length within a 10 km circle around each cell divided by circle
area. These calculations can be done on smaller `tiles' of the study
area to reduce computational requirements and computation time (but has
not been parallelized yet). The magnitude-per-unit area (km/km2) is
calculated by dividing these summed values by pixel area (which makes
m/m2) and multiplied by a factor 1000 to convert them to km/km2. These
road density estimates are used in the \texttt{caribouRSFModel} module.

The \texttt{Buffer} module computes a buffer layer (bufferMap) of
buffers with a specific radius around a set of features from the
anthrDisturb object. The buffer radius can be varied but is 250m here.

The default raster identifies all cells that are covered by the buffer
layer. Buffered cells have a value of 1, other cells have a value of 0.
These raster computations can be done for by feature class, defining a
raster brick of up to seven layers, or aggregated into a cummulative
dustrubance later, as here. The resultant object, bufferMap, is masked
to waterRaster, removing road density values in cell that represent
water. bufferMap is used in the \texttt{caribouPopGrowth} module.

\emph{Important limitations of the \texttt{anthropogenic} module} 1. All
linear features are created at initialisation, regardless of feature
length.

\begin{enumerate}
\def\labelenumi{\arabic{enumi}.}
\setcounter{enumi}{1}
\item
  Anthropogenic features are fixed. They do not deteriorate, recover or
  become otherwise restored over time. This is problematic for seismic
  lines, especially.
\item
  The adeqaucy of the Fast Fourier Trasnform approximation used in
  module \texttt{lineDensity} has not been verifed.
\item
  Only spatial line objects are considered in defining bufferMap. Point
  and polygon objects are neglecetd.
\end{enumerate}

Future versions on the \texttt{anthropogenic} module could include: 1.
Specific phases of linear feature development that are scheduled within
the simulation (to address assumption \#1).

\begin{enumerate}
\def\labelenumi{\arabic{enumi}.}
\setcounter{enumi}{1}
\item
  Deterioration or recovery of anthropogenic features through time (to
  address assumption \#2)
\item
  Inclusion of other anthropogenic features (e.g.~development of
  forestry roads, new seismic lines, etc.). This option could include a
  link to a future module simulating timber harvest that keeps track of
  cut-block location and age. A review of preliminary forest harvesting
  modules written by Cumming for use in SpaDES teaching found that they
  were not ready to be adapted to a situation with multiple yield
  curves; also the cut-block layout and sequencing algorithms are too
  slow to run over entre FMAs. Some development work is required, which
  is planned for the summer.
\item
  Interaction between \texttt{LandR} and this \texttt{anthropogenic}
  module so that pixels that intersect with disturbance features can be
  classified as disturbed in vegetation succession models, or so that
  the proportional area of a pixel that is disturbed can be accounted
  for in biomass calculations.
\item
  Enable road replacement where new roads are `constructed', e.g. (parts
  of) new roads may `replace' already existing roads. Adding these
  features, without replacing old features where they overlap results in
  an overestimation of road densities.
\item
  replace the FFT approximation with external python code to emulate
  ArcMap's lineDensity function (sensu Environment Canada 2011, Appendix
  7.3).
\item
  Incorporation of spatial objects other than spatial lines (to address
  assumption \#4).
\item
  An output could be provided to calculate the percentage of a cell
  within the BufferMap that that is covered by the buffer layer of each
  feature class.
\item
  The ability to vary the buffer widths by feature class, which would
  allow testing of the effect of buffer sizes on caribou recruitment and
  population growth rate.
\end{enumerate}

\textcolor{blue}{birdsNWT}: This module projects species abundance of
bird species under forecasted landscape and climate conditions. The
module develops boosted regression tree (BRT) models from avian
point-count data and associated covariates using the gbm.step function
in the `dismo' package (Hijmans et al. 2011) with a Poisson distribution
and 10-fold cross-validation. BRT settings are as recommended by Elith
et al. (2008) and are consistent with Stralberg et al. (2015). The
module includes a `brtplot' function that, for a given BRT model object,
summarizes cross-validation statistics and variable importance (csv
output), generates partial dependence plots (pdf output), generates
spatial predictions from input data layers (geotiff format), and
generates maps of spatial predictions (png format). The `cvsum' function
summarizes cross-validation statistics across species. Individual
species results are posted here:
(\url{https://drive.google.com/drive/folders/1cpt-AKDbnlUEi6r70Oow2lEPrbzQfVpt?usp=sharing})

Avian data for Bird Conservation Region (BCR) 6 were extracted from the
Boreal Avian Modelling project avian dataset (Cumming et al. 2010,
Barker et al. 2015) and supplemented with data from automated recording
units (ARU) collected by Environment and Climate Change Canada (Haché et
al. unpublished data) and the University of Alberta Bioacoustic Unit
(Bayne et al. unpublished data). Density offsets were calculated
according to methods described in Sólymos et al. (2013), and were based
on the assumption that ARU data detectability is similar to
detectability by human observers (Yip et al. 2017).

Bird models were based on a restricted set of covariates that were
either (a) assumed static over time or (b) direct outputs of the
\texttt{LandR} biomass module. We intentionally excluded climate
covariates that also drive changes in vegetation and fire, given the
assumed lag in vegetation response to climate, and associated
statistical decoupling of these covariates over time (Figure 1).

Model covariates assumed static through simulation periods included the
following landcover covariates derived from the Comission for
Environmental Cooperation (CEC) North American Landcover 2005:

WAT = water (1/0) URBAG = urban/agriculture (1/0) LED25 = water
proportion within 5x5 moving window (continuous) DEV25 = development
proportion within 5x5 moving window (continuous)

We also included a categorical landform covariate derived from
AdaptWest's land facet layer (Michalak et al. 2015), with the following
class definitions:

1 Valley 2 Hilltop in Valley 3 Headwaters 4 Ridges and Peaks 5 Plains 6
Local Ridge in Plain 7 Local Valley in Plain 8 Gentle Slopes 9 Steep
Slopes

Model covariates allowed to vary according to simulation outputs were
tree species biomass estimates (g/m2) derived from Beaudoin et al.'s
(2017) predicted biomass layers for 2001 or 2011, depending on the date
of survey (pre-2006 was associated with 2001 vegetation; 2006 and later
was associated with 2011 vegetation). Biomass covariates for the
following speices were included:

\emph{Abies balsamea, Betula papyrifera, Larix laricina, Picea glauca,
Picea mariana, Pinus banksiana, Pinus contorta, Populus balsamifera,
Populus tremuloides}

Models were developed for 71 passerine and near-passerine species that
a) currently breed either in the Northwest Territories or Alberta
portions of BCR 6, b) for which density offsets were available, and c)
for which data were sufficient to fit cross-validated BRTs (see list
below). For this project simulations were generated for 10 species,
including:

OVEN Ovenbird (\emph{Seiurus aurocapilla}) AMRE American Redstart
(\emph{Setophaga ruticilla}) BLPW Blackpoll Warbler (\emph{Setophaga
striata}) PAWA Palm Warbler (\emph{Setophaga palmarum}) CAWA Canada
Warbler (\emph{Cardellina canadensis}) FOSP Fox Sparrow
(\emph{Passerella iliaca}) WCSP White-crowned Sparrow (\emph{Zonotrichia
leucophrys}) RUBL Rusty Blackbird (\emph{Euphagus carolinus}) RCKI
Ruby-crowned Kinglet (\emph{Regulus calendula}) OSFL Olive-sided
Flycatcher (\emph{Contopus cooperi})

\emph{Important assumptions of the \texttt{birdsNWT} module} 1. The
relative abundance of bird species is based on an environmental niche
modeling approach with associated assumptions (Wiens et al.
2009).variation in occurrence, rather than variations in movement
(Stewart et al. 2018; Neilson et al. 2018).

\begin{enumerate}
\def\labelenumi{\arabic{enumi}.}
\setcounter{enumi}{1}
\item
  Density offsets assume that ARU data detectability is similar to
  detectability by human observers (sensu Yip et al. 2017).
\item
  We assume a lag in vegetation response to climate and therefore do not
  include climate covariates within this module.
\end{enumerate}

Future versions of the \texttt{birdsNWT} module could include: 1.
Incorporation of additional covariates, such as linear feature density.

\begin{enumerate}
\def\labelenumi{\arabic{enumi}.}
\setcounter{enumi}{1}
\tightlist
\item
  Simulation for the remaining 61 species of passerine and
  near-passerine birds that breed either in the NWT or Alberta portions
  of BCR6. A full list of 71 possible species include:
\end{enumerate}

Alder Flycatcher (\emph{Empidonax alnorum}), American Crow (\emph{Corvus
brachyrhynchos}), American Goldfinch (\emph{Spinus tristis}), American
Pipit (\emph{Anthus rubescens), American Redstart (}Setophaga
ruticilla\emph{), American Robin (}Turdus migratorius\emph{), American
Tree Sparrow (}Spizella arborea\emph{), Black-and-white Warbler
(}Mniotilta varia\emph{), Bay-breasted Warbler (}Setophaga
castanea\emph{), Black-backed Woodpecker (}Picoides arcticus\emph{),
Black-capped Chickadee (}Poecile atricapillus\emph{), Brown-headed
Cowbird (}Molothrus ater\emph{), Blue-headed Vireo (}Vireo
solitarius\emph{), Blackburnian Warbler (}Setophaga fusca\emph{), Blue
Jay (}Cyanocitta cristata\emph{), Blackpoll Warbler (}Setophaga
striata\emph{), Boreal Chickadee (}Poecile hudsonicus\emph{), Brewer's
Blackbird (}Euphagus cyanocephalus\emph{), Brown Creeper (}Certhia
americana\emph{), Brown Thrasher (}Toxostoma rufum\emph{), Black
throated Green Warbler (}Setophaga virens\emph{), Canada Warbler
(}Cardellina canadensis\emph{), Clay colored Sparrow (}Spizella
pallida\emph{), Cedar Waxwing (}Bombycilla cedrorum\emph{), Chipping
Sparrow (}Spizella passerina\emph{), Cape May Warbler (}Setophaga
tigrina\emph{), Common Grackle (}Quiscalus quiscula\emph{), Connecticut
Warbler (}Oporornis agilis\emph{), Common Raven (}Corvus corax\emph{),
Common Redpoll (}Acanthis flammea\emph{), Common Yellowthroat
(}Geothlypis trichas\emph{), Chestnut-sided Warbler (}Setophaga
pensylvanica\emph{), Dark-eyed Junco (}Junco hyemalis\emph{), Eastern
Kingbird (}Tyrannus tyrannus\emph{), Eastern Phoebe (}Sayornis
phoebe\emph{), Evening Grosbeak (}Coccothraustes vespertinus\emph{), Fox
Sparrow (}Passerella iliaca\emph{), Golden crowned Kinglet (}Regulus
satrapa\emph{), Gray-cheeked Thrush (}Catharus minimus\emph{), Gray Jay
(}Perisoreus canadensis\emph{), Gray Catbird (}Dumetella
carolinensis\emph{), Hammond's Flycatcher (}Empidonax hammondii\emph{),
Hermit Thrush (}Catharus guttatus\emph{), Horned Lark (}Eremophila
alpestris\emph{) , House Wren (}Troglodytes aedon\emph{), Le Conte's
Sparrow (}Ammodramus leconteii\emph{), Least Flycatcher (}Empidonax
minimus\emph{), Lincoln's Sparrow (}Melospiza lincolnii\emph{), Magnolia
Warbler (}Setophaga magnolia\emph{), Mourning Warbler (}Geothlypis
philadelphia\emph{), Nashville Warbler (}Oreothlypis ruficapilla\emph{),
Northern Waterthrush (}Parkesia noveboracensis\emph{), Orange-crowned
Warbler (}Oreothlypis celata\emph{), Olive-sided Flycatcher (}Contopus
cooperi\emph{), Ovenbird (}Seiurus aurocapilla\emph{), Palm Warbler
(}Setophaga palmarum\emph{), Philadelphia Vireo (}Vireo
philadelphicus\emph{), Pine Grosbeak (}Pinicola enucleator\emph{), Pine
Siskin (}Spinus pinus\emph{), Pileated Woodpecker (}Dryocopus
pileatus\emph{), Purple Finch (}Haemorhous purpureus\emph{),
Rose-breasted Grosbeak (}Pheucticus ludovicianus\emph{), Red-breasted
Nuthatch (}Sitta canadensis\emph{), Ruby-crowned Kinglet (}Regulus
calendula\emph{), Red-eyed Vireo (}Vireo olivaceus\emph{), Ruffed Grouse
(}Bonasa umbellus\emph{), Rusty Blackbird (}Euphagus carolinus\emph{),
Red-winged Blackbird (}Agelaius phoeniceus\emph{), Savannah Sparrow
(}Passerculus sandwichensis\emph{), Sedge Wren (}Cisthothorus
platensis\emph{), Song Sparrow (}Melospiza melodia\emph{), Swamp Sparrow
(}Melospiza georgiana\emph{), Swainson's Thrush (}Catharus
ustulatus\emph{), Tennessee Warbler (}Oreothlypis peregrina\emph{),
Townsend's Warbler (}Setophaga townsendi\emph{), Tree Swallow
(}Tachycineta bicolor\emph{), Varied Thrush (}Ixoreus naevius\emph{),
Veery (}Catharus fuscescens\emph{), Vesper Sparrow (}Pooecetes
gramineus\emph{), Warbling Vireo (}Vireo gilvus\emph{), White-breasted
Nuthatch (}Sitta carolinensis\emph{), White-crowned Sparrow
(}Zonotrichia leucophrys\emph{), Western Tanager (}Piranga
ludoviciana\emph{), Western Wood-Pewee (}Contopus sordidulus\emph{),
Wilson's Warbler (}Cardellina pusilla\emph{), Winter Wren (}Troglodytes
hiemalis\emph{), White-throated Sparrow (}Zonotrichia albicollis\emph{),
White-winged Crossbill (}Loxia leucoptera\emph{), Yellow-bellied
Sapsucker (}Sphyrapicus varius\emph{), Yellow-bellied Flycatcher
(}Empidonax flaviventris\emph{), Yellow-rumped Warbler (}Setophaga
coronata\emph{), Yellow Warbler (}Setophaga petechia*)

\textcolor{blue}{caribou}: The \texttt{caribou} module is composed of
two main modules: \texttt{caribouRSFModel} and
\texttt{caribouPopGrowthModel}. Each module loads existing glms
developped by Environment Canada 2011 to project caribou occurrence and
demographics over 100 years as the anthropogenic (\texttt{anthropogenic}
module), fire (\texttt{FireSense} module), and natural features
(\texttt{LandR} module) of the NWT BCR6 region (i.e.~Taiga plains)
changes. Combined, these modules incorporate existing statistical
relationships, with simulated landscape predictions, to produce the
probability of occurrence, and population growth rate, of Woodland
caribou (boreal population) within the NWT.

\texttt{caribouRSFModel} The statistical relationship between caribou
telemetry collars and the landscape has been summarized within the
Environment Canada 2011 Scientific Report as a resource selection
function (sensu Manly et al. 2002). This module is based off of the top
Taiga Plains ecozone resource selection function (Environment Canada
2011; Table 46), generated from data collected from 24 Alberta, 50
British Columbia, and 169 NWT collared adult female caribou between 2000
and 2010 (Environment Canada 2011; Table 39).

In the Taiga Plains ecozone, the relative probability of adult female
boreal caribou resource selection (0 to 1) is best described by
(Environment Canada 2011; Table 46):

\(Relative habitat selection ~ Elevation + Elevation^2 + Vrug + Vrug^2 + RoadDensity + Deciduous + Shrub + Eq 1.  Herb + Water + RecentBurn + OldBurn\)

The definition, and units of measurement, for each of the predictor
variables in this relationship can be found in Environment Canada 2011
Table 40. The methods behind model development can be found throughout
Environment Canada 2011 Appendix 7.3.

Model covariates were assumed to either be static, or were obtained from
other modules within this project: 1 Elevation (static; AdaptWest) 2
Vrug (static; adapted from AdaptWest sensu Sappington et al. 2007) 3
RoadDensity (static; \texttt{anthropogenic}) 4 Deciduous (dynamic;
\texttt{LandR}) 5 Shrub (static; LCC05) 6 Herb (static; LCC05) 7 Water
(static; LCC05) 8 RecentBurn (dynamic; \texttt{FireSense}) 9 OldBurn
(dynamic; \texttt{FireSense})

This module determines the relative habitat selection of caribou based
on each of the above habitats on a yearly time step. This module
produces a rasterized map of this relative selection across the NT1
Boreal Caribou Range (i.e.~BCR6 within the NWT) for every 10 years of
the simulation. Options include outputting maps for range planning
regions outlined in either Figure 2 and/or Figure 3.

\emph{Important assumptions of the \texttt{caribouRSFModel} module} 1.
Caribou are distributed in an ideal free distribution across the Taiga
plains (Fretwell and Lucas 1970).

\begin{enumerate}
\def\labelenumi{\arabic{enumi}.}
\setcounter{enumi}{1}
\item
  Resource selection leads to increased population fitness and
  population density (Manly et al. 2002). The resource selection
  documented here is a relative probability, rather than true
  probability, as we used a use-available design (Manly et al. 2002).
\item
  The Taiga ecozone RSF model developed by Environment Canada 2011
  reliably predicts the relative probability of occurrence for woodland
  boreal caribou. See section 6.2.4 of Environment Canada 2011 for
  reasons this assumption may be violated.
\item
  Shrub and Herbacious landcover remains constant across the simulation
  period (2010 to 2110) despite the impacts of fire, succession, or
  other landscape change processes.
\item
  Caribou VHF collar data is not biased - caribou were randomly, and
  opportunistically, collared. Caribou were not collared for monitoring
  impacts of natural resource development projects, or other motives.
\item
  Caribou do not display season-specific habitat selection; the RSF
  model used here is based off of annual caribou ccurrences. It does not
  account for seasonal variations.
\item
  OldBurns are considered to be anything older than 40 years. We assume
  that OldBurns are between 41 and 93 years old, as this would have been
  the maximum time since fire that was quantifiable in Environment
  Canada 2011 (Appendix 7.3, section 5.3;1917 to 2010).
\end{enumerate}

Future versions of the \texttt{caribouRSFModel} module could
incorporate; 1. Incorporation of the national realized caribou RSF model
(Environment Canada 2011; Table 43) which has higher predictive capacity
than the Taiga Plains RSF modeled here.

\begin{enumerate}
\def\labelenumi{\arabic{enumi}.}
\setcounter{enumi}{1}
\tightlist
\item
  Future model outputs could be based off of the regional planning units
  specified in: {[}Figure 2. Boreal caribou NT1 range planning regions
  2018{]}
  (\url{https://github.com/tatimicheletti/NWT/blob/master/Boreal_caribou_NT1_range_planning_regions_2018.jpg})
  {[}Figure 3. Units for ECCC modeling project 2019{]}
  (\url{https://github.com/tati-micheletti/NWT/blob/master/Units_for_ECCC_modeling_project_2019.jpg})
  The outputs could involve both relative selection, and uncertainty of
  the relative selection (i.e.~standard deviation of the mean relative
  selection at each pixel) maps.
\end{enumerate}

\texttt{caribouPopGrowthModel} The \texttt{caribouPopGrowthModel} module
builds upon the statistical relationship underpinning today's national
boreal caribou recovery strategy (Environment Canada 2011; 2012). The
module recalculates the response variable of this relationship (caribou
calf recruitment) on a yearly time interval, and uses these data to
determine the population growth rate, lambda (λ).

Specifically, boreal caribou recruitment (Rec) - the proportion of
caribou cows:calves at winter survey periods - is best predicted by the
total disturbance within a caribou study area (Environment Canada 2011;
Table 55, model M3):

\begin{verbatim}
                                      $Rec ~ total disturbance$                                     Eq 2.
\end{verbatim}

The methods behind Eq 2. development can be read in Environment Canada
2011 (Appendix 7.6). The total disturbance within a study area is
described as the ``percent total non-overlapping fire and anthropogenic
disturbance (500 m buffer on anthropogenic; reservoirs removed)''
(Environment Canada 2011, Table 54). Fire is measured as ``percent fire
\textgreater{} 40 years old'' within a caribou study area and
anthropogenic is measured as ``percent non-overlapping anthropogenic
disturbance (500 m buffer)'' within a caribou study area (Environment
Canada 2011; Table 52). This module uses the rasterized output maps of
both the \texttt{FireSense} (climate sensitive), and
\texttt{anthropogenic} module. These map the presence/absence of fire
\textless{} 40 years old, or anthropogenic disturbances, at each 250 m x
250m pixels. These rasterized maps were over-layed, were used to
calculate the percent non-overlapping total disturbance within a caribou
study area for each time step, and are used as inputs for this
statistical relationship (Eq 2.).

This module projects Eq 2 based off the the coefficients provided in
(Environment Canada 2011; Table 56, model M3).

This module uses annually calculated recruitment values to determine the
caribou population growth rate - the proportional change in population
size (lambda; λ) - for each polygon in {[}Figure 2{]} and {[}Figure
3{]}. The population model used to estimate lambda on a yearly time
interval is well published for both caribou and other ungulates (Hatter
and Bergerud 1991; McLoughlin et al. 2003; Sorensen et al. 2006;
Hervieux et al. 2014):

\begin{verbatim}
                                        $λ = (1 - M)/(1 - Rec)$                 Eq 3.
\end{verbatim}

Where recruitment (Rec) only involves juvenile female caribou (and
assumes a 50:50 sex ratio), and adult female mortality (M) is calculated
as 1 - adult female survival (SadF). At each yearly time step λ is
saved. This module outputs a plot of lambda, for each requested region
or caribou planning area {[}Figure 2{]}, {[}Figure 3{]}, over the
simulated time interval (100 years; 2010 through 2110). Please see the
below assumptions for important information regarding these output
plots.

\emph{Important assumptions of the \texttt{caribouPopGrowthModel}
module} 1. The statistical relationship between recruitment and the
landscape described by Environment Canada is the correct relationship
for all Northwest Territories boreal caribou local population units.
(Model M3, EC 2011 Table 56). This relationship was determined from 24
boreal caribou study areas across Canada (N = 24), and lacks information
on study areas primarily driven by fire (EC 2011, Figure 67).

\begin{enumerate}
\def\labelenumi{\arabic{enumi}.}
\setcounter{enumi}{1}
\item
  Adult female survival is constant. We used the Environment Canada 2011
  recommended value of 0.85. We assume that this value applies to all
  populations of boreal woodland caribou in the Northwest Territories,
  and that it does not vary across the projected time of these modules.
\item
  We assume that caribou are homogeneously distributed throughout the
  requested regions and caribou planning areas of the NWT (i.e.~the
  `Proposed caribou Range Planning Regions' and requested `Units for
  ECCC modeling projects').
\item
  We assume that caribou do not move between the requested regions and
  planning areas of the NWT (i.e.~the `Proposed caribou Range Planning
  Regions' and requested `Units for ECCC modeling projects'); we assume
  these areas are functionally isolated from one another.
\end{enumerate}

Future versions of the \texttt{caribouPopGrowthModel} module could
address many of the above assumptions. An updated module could
incorporate; 1. Incorporation of seismic lines as part of the
\texttt{anthropogenic} module to ensure that the effect of future
seismic lines affects caribou recruitment, as assumed in Environment
Canada 2011.

\begin{enumerate}
\def\labelenumi{\arabic{enumi}.}
\setcounter{enumi}{1}
\tightlist
\item
  Revised caribou study areas (or local population units - the
  recommended scale of population management; Environment Canada 2012),
  demographic values, and/or statistical models to incorporate
  contemporary data under updated data sharing agreements.
\end{enumerate}

3a. Flexibility to test multiple top statistical models presented in
Environment Canada 2011 ( i.e.~rather than solely Eq 3.; Table 55, Table
56) or future reports.

3b. The output of the \texttt{caribouRSFModel} as a predictor in the
\texttt{caribouPopGrowthModel} statistical model. This would explicitly
link these two modules by using high quality habitat as a predictor of
caribou recruitment (sensu Environment Canada 2011, Table 55, model
M12).

\begin{enumerate}
\def\labelenumi{\arabic{enumi}.}
\setcounter{enumi}{3}
\item
  A statistical relationship between recruitment and landscape change
  that is specific to the NWT (addressing assumption \# 1 above).
\item
  A statistical relationship between adult female survival and landscape
  change that is specific to the NWT (addressing assumption \# 2 above).
\item
  Incorporation of a caribou population model that is both logistic
  (i.e.~density dependent) and/ or stochastic to look at projected
  variations in population size.
\end{enumerate}

\paragraph{\texorpdfstring{Outputs from
\texttt{caribouPopGrowthModel}}{Outputs from caribouPopGrowthModel}}\label{outputs-from-cariboupopgrowthmodel}

\paragraph{Figure: Annual plot of projected lambda across each requested
study region of the
NWT.}\label{figure-annual-plot-of-projected-lambda-across-each-requested-study-region-of-the-nwt.}

\paragraph{Figure: Annual plot of projected lambda uncertainty across
each requested study region of the NWT. Uncertainty is measured for each
study region as the standard deviation of the projected lambda
values.}\label{figure-annual-plot-of-projected-lambda-uncertainty-across-each-requested-study-region-of-the-nwt.-uncertainty-is-measured-for-each-study-region-as-the-standard-deviation-of-the-projected-lambda-values.}

\section{Intgrated modules components of Item 3: Community metrics, and
an analysis of the
Edehzhie}\label{intgrated-modules-components-of-item-3-community-metrics-and-an-analysis-of-the-edehzhie}

\textcolor{blue}{comm_metricsNWT}: This module computes a series of
avian community metrics based on the suite of species-level predicted
abundance maps provided from the \texttt{birdsNWT} module. These
community metrics were calculated on a 10 year interval. This module
outputs a raster for each diversity metric of the current simulation
year, along with a table of mean value for the Shannon diversity index,
Simpson diversity index, and Rao's quadratic entropy metrics.

There are four community metrics computed as part of the
\texttt{comm\_metricsNWT} module:

\begin{enumerate}
\def\labelenumi{\arabic{enumi}.}
\tightlist
\item
  Expected Species Richness: Expected Species Richness was estimated as
  the probability of occupancy based on the poisson distribution:
\end{enumerate}

\[ S: 1 - \exp^{\lambda}\]\\
where: \[ exp^{-\lambda} = \frac{exp^{-\lambda}\lambda^k{}{k!} \]

\begin{enumerate}
\def\labelenumi{\arabic{enumi}.}
\setcounter{enumi}{1}
\tightlist
\item
  Shannon diversity index: This index assumes that individuals are
  randomly sample from an infinitely large community, and that all
  species are represented in the sample (Magurran 2004).
\end{enumerate}

\[ H^' = - \sum_{i=1}^{S} p_ilnp_i \] where: \[p_i\] is the proportion
of individuals found in the \emph{ith} species.

\begin{enumerate}
\def\labelenumi{\arabic{enumi}.}
\setcounter{enumi}{2}
\tightlist
\item
  Simpson diversity index: This index estimates the probability of any
  two individuals drawn at randoms from an infinitely large community
  belonging to the same species (Magurran 2004).
\end{enumerate}

\[ 1-D = \sum p_i^2 \] where:\[p_i\] is the proportion of individuals
found in the \emph{ith} species.

Simpson's diversity has more ecological significance than Shannon index,
because it represents the probability that two individuals of the same
species meet.

\begin{enumerate}
\def\labelenumi{\arabic{enumi}.}
\setcounter{enumi}{3}
\tightlist
\item
  Rao's Quadratic Entropy:
\end{enumerate}

\[ \sum_{i=1}^{S-1} \sum_{j=i+1}^{S} d_ij* p_i* p_i\]

Where: \[d_{ij}\] is the difference between the \emph{ith} and
\emph{jth} species. This difference is is estimated using functional
attributes\for each species. In addition, Raos Quadratic Entropy
incorporates two elements: 1. relative abundances of species, and 2. a
measure of pairwise functional differences between species (Botta-Dukat
2005). For this second element, we used Life history traits as
considered in Knaggs (2018).

\emph{Current limitations of the \texttt{comm\_metricsNWT} module} 1.
Species richness and diversity indices summarize information about the
relative abundance of species within a community but ignore the degree
of difference between species. This limitation has been criticized
because this measures does not take into account the uniqueness or
ecological importance of individual species.

\begin{enumerate}
\def\labelenumi{\arabic{enumi}.}
\setcounter{enumi}{1}
\tightlist
\item
  Rao's Entropy depends on delineating functional groups within an
  significant ecological sense. Functional traits must be selected
  carefully.
\end{enumerate}

Future versions of the \texttt{comm\_metricsNWT} module could include:
1. Additional species richness estimates (e.g.~Cowel, Clench , or Chao
1; Gotelli and Colwell 2011)

\begin{enumerate}
\def\labelenumi{\arabic{enumi}.}
\setcounter{enumi}{1}
\item
  Outputs for each community metric that are specific to different study
  areas (e.g.~specific to different caribou study areas).
\item
  Species turnover based on community dissimilatiry that considers
  different climate change scenarios (sensu Stralberg et al. 2009)
\end{enumerate}

\textcolor{blue}{Edehzhie}: For the third deliverable, we have repeated
the analysis developed for the whole BCR6 within NWT territory described
above, but only within the proposed area of Edehzhie. Bird species
abundance (\texttt{birdsNWT}), caribou population growth
(\texttt{caribouPopGrowthModel}), and caribou and relative habitat
selection (\texttt{caribouRSFModel}) predictions from this polygon
(area) were compared to predictions on three similar areas randomly
located north, west, and south of the Edehzhie area. These results are
presented below.

\paragraph{Figure: One bird species example, 10 plots of prediction for
the three areas one next to the other (gif for html/all other
species)}\label{figure-one-bird-species-example-10-plots-of-prediction-for-the-three-areas-one-next-to-the-other-gif-for-htmlall-other-species}

\paragraph{Figure: Caribou population growth on the three
areas}\label{figure-caribou-population-growth-on-the-three-areas}

\paragraph{Figure: RSF over the three
areas}\label{figure-rsf-over-the-three-areas}

\paragraph{Table: Summarized RSF mean
values}\label{table-summarized-rsf-mean-values}

\section{Intgrated modules final component of Item 3: Species
co-occurrence}\label{intgrated-modules-final-component-of-item-3-species-co-occurrence}

As requested deliverable of this project was to conduct species
co-occurrence analyses for bird Species At Risk to evaluate potential
conservation claues of multi-species conservation planning throughout
the simulation period. Results from this type of anlysis could help
inform land use planning and conservation efforts in the study area.

Future versions of this project could incorporate this species
co-occurrence deliverable one the other module components are fully
tested. This will be completed for a manuscript, and is subject to (1)
greater precision of the above mentioned modules, and the finalized set
of bird Species At Risk for which models can be fit (see
\texttt{birdsNWT}).

To best perform this analysis we require more information on the
objective of the analysis. For example, how is multi-species co
occurence of birds and caribou to be measured, and over what spatial
units?

\section{Concluding remarks}\label{concluding-remarks}

\subsection{From data to decisions in SpaDES: an integrated projection
project for the Northwest
Territories}\label{from-data-to-decisions-in-spades-an-integrated-projection-project-for-the-northwest-territories}

This project represents a first in data integration and decision making
of natural resources; it combines state of the art models from multiple
ecological fields over an extensive spatial area to forecast vertebrate
ecology. As such, the project represents a powerful tool for natural
resource decition making, but also demonstrates the intensive
collaborative ability of ecologists to work simultaneously on a national
project. Finally, the project is open source and repeatable - the basic
requirements to hold science accountable and a must for the, sometimes
controversial, natural resource desitions required both today and in
future.

Recovering Species at Risk is mandated nationally under the federal
Species at Risk Act, and internationally under the United Nationas
Convention on Biological Diversity Aichi Target 1. However, the ideal
location and strategies required to recover species may vary with
spatial and temporal landscape change. These strategies may also require
tradeoffs in natural resource development such as altered anthropogenic
landuses (mining, oil and gas, forestry) with conservation strategies
(such as protected areas). The ideal location and configuration of such
spatial tradeoffs are best evaluated if we can anticipate landscape
change, and the effect of this change on species abundance and
demography, in the future. Here, we take a first step to integrating all
of this information,and provide relecavant results for boreal bird, and
woodland boreal caribiou, species - listed under the Species at Risk
Act. The structure of this integrative project could be repeated for
other Species at Risk, and represents an important step towards species
conservation considering future, rather than current or past, landscape
considerations.

Ensuring science is repeatable, adaptable, and open allows applied
disciplines to ensure they progress under an adaptive management
framework. Discrete event simulation modeling that is scheduled, through
platforms such as SpaDES, ensure that assumptions can be built upon in
an open and repeatable way that enhances projections through time.
Within this project we have provided the required bird species abundance
models, caribou occurence, and caribou population growth rates using
SpaDES to ensure that the project can continually improve, the
projections can be refined, and natural resouce decisions can be adapted
to anticipate policy, economic, ecological, or climate driven changes
that are bound to crop up over time. The reported project represents the
cutting edge in integrated decision making, with ramification for
Canada's national and international policy commitments to our natural
resources.

\section{Literature cited}\label{literature-cited}

Araújo, M. B., W. Thuiller, P. H. Williams and I. Reginster. 2005.
Downscaling European species atlas distributions to a finer resolution:
implications for conservation planning. Global Ecology and Biogeography,
14(1):17-30.

Armstrong, G. and S. Cumming. 2003. Estimating the cost of land base
changes due to wildfire using shadow prices. Forest Science 49(5):
719--730.

Barker, N. K. S., P. C. Fontaine, S. G. Cumming, D. Stralberg, A.
Westwood, E. M. Bayne, P. Sólymos, F. K. A. Schmiegelow, S. J. Song, and
D. J. Rugg. 2015. Ecological monitoring through harmonizing existing
data: Lessons from the boreal avian modelling project. Wildlife ociety
Bulletin 39:480-487. \url{http://dx.doi.org/10.1002/wsb.567}

Barros, C., Y. Lou, E. J. B. McIntire, A. M. Chubaty, I. Eddy, D.
Andison and S. Cumming. Empowering ecologists in a simulation context:
using R data for complex landscape modelling, the LandR study case. In
preparation.

Beaudoin, A., P. Y. Bernier, P. Villemaire, L. Guindon, and X. J. Guo.
2017. Tracking forest attributes across Canada between 2001 and 2011
using a k nearest neighbors mapping approach applied to MODIS imagery.
Canadian Journal of Forest Research 48:85-93.
\url{http://dx.doi.org/10.1139/cjfr-2017-0184}

Bergeron, Y., D. Dominic, M. P. Girardin, and C. Carcaillet. 2010. Will
climate change drive 21st century burn rates in Canadian boreal forest
outside of its natural variability: collating global climate model
experiments with sedimentary charcoal data. International Journal of
Wildland Fire 19:1127-1139. doi: 10.1071/WF09092

Botta-Dukát, Z. 2005. Rao's quadratic entropy as a measure of functional
diversity based on multiple traits. Journal of Vegetation Science
16:533-540.

Burton, A. C., F. E. C. Stewart, and J. T. Fisher. Scaling-down species
distribution models: can large-scale models predict smaller scale
wildlife distribution? In Review.

Convention on Biological Diversity 2020. 2020 Biodiversity Goals and
Targets for Canada. Accessed March 28, 2019.
(\url{https://www.canada.ca/en/parks-canada/news/2016/12/2020-biodiversity-goals-targets-canada.html})

Chen, T. and C. Guestrin. 2016. XGBoost: A Scalable Tree Boosting
System. In 22nd SIGKDD Conference on Knowledge Discovery and Data
Mining, 2016. \url{http://arxiv.org/abs/1603.02754}

Chubaty, A. M. and E. J. B. McIntire. 2017. SpaDES: Develop and Run
Spatially Explicit Discrete Event Simulation Models. R package version,
2(0).

Cumming, S. G., D. Demarchi and C. Walters. 1998. A Grid-Based Spatial
Model of Forest Dynamics Applied to the Boreal Mixedwood Region. Working
Paper 1998--8. Sustainable Forest Management Network.
\url{https://doi.org/10.7939/R35N33}.

Cumming, S. G., K. L. Lefevre, E. Bayne, T. Fontaine, F. K. A.
Schmiegelow, and S. J. Song. 2010. Toward conservation of Canada's
boreal forest avifauna: design and application of ecological models at
continental extents. Avian Conservation and Ecology 5(2):8.

Elith, J., J. R. Leathwick, and T. Hastie. 2008. A working guide to
boosted regression trees. Journal of Animal Ecology 77:802-813.

Environment Canada. 2012. Recovery Strategy for the Woodland Caribou
(\emph{Rangifer tarandus caribou}), Boreal population, in Canada.
Species at Risk Act Recovery (138 pp.) Accessed online August 10, 2018
at
\url{http://publications.gc.ca/collections/collection_2012/ec/En34-140-2012-eng.pdf}.

Environment Canada. 2011. Scientific Assessment to Inform the
Identification of Critical Habitat for Woodland Caribou (\emph{Rangifer
tarandus caribou}), Boreal Population, in Canada: 2011 Update. Ottawa,
Ontario, Canada. 102 pp.~plus appendices.

Environment Canada. 2008. Scientific Review for the Identification of
Critical Habitat for Woodland Caribou (\emph{Rangifer tarandus
caribou}), Boreal Population, in Canada. August 2008. Ottawa:
Environment Canada. 72 pp.~plus 180 pp Appendices.

Fretwell, S. D., and J. H. J. Lucas. 1970. On territorial behavior and
other factors influencing habitat distribution in birds. Acta
Biotheoretica 19:16--36.

Gotelli, N. and R. Colwell. 2011. Estimating species richness.
\emph{In}: Magurran, A. and B. McGill.2011. Biological diversity:
frontiers in measurement and assessment. Oxford University Press. Pp:
39- 54.

Government of the Northwest Territories. 2019. A Framework for Boreal
Caribou Range Planning: Revised Draft - Appendices. Environment and
Natural Resources, Government of the Northwest Territories, Yellowknife,
NT. 38 + ii pp.

Hargrove W. W., R. H. Gardner, M. G. Turner, W. H. Romme, D. G. Despain.
2000. Simulating fire patterns in heterogeneous landscapes. Ecological
Modelling 135:243--63.

Hatter, I. W. and W. A. Bergerud. 1991. Moose recruitment, adult
mortality and rate of change. Alces 27:65-73.

Hervieux, D., M. Hebbelwhite, D. Stepinsky, S. Bacon and S. Boutin.
2014. Managing wolves (\emph{Canis lupus}) to recover threatened
woodland caribou (\emph{Rangifer tarandus caribou}) in Alberta.
Cananadian Journal of Zoology 92(12):1029-1037. doi:
10.1139/cjz-2014-0142

Hijmans, R. J., S. Phillips, J. Leathwick, and J. Elith. 2011. Package
`dismo'. Available online at \url{http://cran.r}
project.org/web/packages/dismo/index.html.

Holling, C. S. 1992. Cross‐scale morphology, geometry, and dynamics of
ecosystems. Ecological monographs 62(4): 447-502.

Knagss, M. 2018. Effects of burn severity and time since fire on
songbird communities in the northern boreal forest. Department of
Renewable Resources. University of Alberta. Master's Thesis.

Latifovic, R. and D.Pouliot. 2005. Multitemporal Land Cover Mapping for
Canada: Methodology and Products. Canadian Journal of Remote Sensing
31(5):347--63. \url{https://doi.org/10.5589/m05-019}.

Levin, S. A. 1992. The problem of pattern and scale in ecology: the
Robert H. MacArthur award lecture. Ecology 73(6):pp.1943-1967.

Magurran, A. 2004. Measuring Biological Diversity. Wiley. 256 pp.~ISBN
0632056339

Manly, B. F. J., L. L. McDonald, D. L. Thomas, T. L. McDonald, and W. P.
Erickson, editors. 2002. Resource selection by animals: statistical
analysis and design for field studies. Second Edition. Kluwer, Boston,
USA.

Mcloughlin, P. D., E. Dzus, B. O. B. Wynes, and S. Boutin. 2003.
Declines in populations of woodland caribou. The Journal of Wildlife
Management 67(4):755-761.

Neilson, E. W., T. Avgar, A. C. Burton, K. Broadley, and S. Boutin.
2018. Animal movement affects interpretation of occupancy models from
camera-trap survey of unmarked animals. Ecosphere 9(1)
\url{doi:10.1002/ecs2.2090}

Pya, N. and S. N. Wood. 2015. Shape Constrained Additive Models.
Statistics and Computing 25(3):543--59.
\url{https://doi.org/10.1007/s11222013-9448-7}.

Riitters, K. H. 2005. Downscaling indicators of forest habitat structure
from national assessments. Ecological indicators 5(4):273-279.

Sappington, J. M., K. M. Longshore and D. B. Thomson. 2007. Quantifiying
Landscape Ruggedness for Animal Habitat Anaysis: A case Study Using
Bighorn Sheep in the Mojave Desert. Journal of Wildlife Management.
71(5):1419 -1426.

Sólymos, P., S. M. Matsuoka, E. M. Bayne, S. R. Lele, P. Fontaine, S. G.
Cumming, D. Stralberg, F. K. A. Schmiegelow, and S. J. Song. 2013.
Calibrating indices of avian density from non-standardized survey data:
making the most of a messy situation. Methods in Ecology and Evolution
4:1047-1058. \url{http://dx.doi.org/10.1111/2041-210x.12106}

Sorensen, T., P. D. McLoughlin, D. Hervieux, E. Dzus, J. Nolan, B. Wynes
and S. Boutin. 2008. Determining Sustainable Levels of Cumulative
Effects for Boreal Caribou. J. Wildlife Manag., 72, 900--905. doi:
10.2193/2007-079

Species At Risk Act. 2002. An Act respecting the protection of wildlife
species at risk in Canada. Accessed March 28, 2019
(\url{https://web.archive.org/web/20040604021711/http://laws.justice.gc.ca/en/S-15.3/text.html})

Stewart, F. E. C., J. T. Fisher, A. C. Burton, and J. P. Volpe. 2018.
Species occurrence data reflect the magnitude of animal movements better
than the proximity of animal space use. Ecosphere 9(2)
\url{doi:10.1002/ecs2.2112}.

Stralberg D., D. Jongsomjit, C. A. Howell, M. A. Snyder , J. D.
Alexander, et al. 2009. Re-Shuffling of Species with Climate Disruption:
A No-Analog Future for California Birds?. PLOS ONE 4(9): e6825.
\url{https://doi.org/10.1371/journal.pone.0006825}

Stralberg, D., S. M. Matsuoka, A. Hamann, E. M. Bayne, P. Sólymos, F. K.
A. Schmiegelow, X. Wang, S. G. Cumming, and S. J. Song. 2015. Projecting
boreal bird responses to climate change: the signal exceeds the noise.
Ecological Applications 25:52--69.
\url{http://dx.doi.org/10.1890/13-2289.1}

Sutherland, G. D., F. K. A. Schmiegelow, C-A. Johnson, E. J. B.
McIntire, M. LeBlond and R. Jagodzinski. A simple empirically-linked
demographic model to estimate likelihoods of recovering population of
boreal caribou (\emph{Rangifer tarandus caribou}). In preparation.

Wiens, J. A. 1989. Scale in ecology. Functional Ecology 3:385-397.

Wiens J. A., D. Stralberg, D. Jongsomjit, C. A. Howell and M. A. Snyder.
2009. Niches, models, and climate change: assessing the assumptions and
uncertainties. Proceedings of the National Academy of Sciences
106(2):19729-36.

Yip, D. A., L. Leston, E. M. Bayne, P. Sólymos, and A. Grover. 2017.
Experimentally derived detection distances from audio recordings and
human observers enable integrated analysis of point count data. Avian
Conservation and Ecology 12.
\url{http://dx.doi.org/10.5751/ACE00997-120111}

\section{Appendix 1}\label{appendix-1}

\subsection{\texorpdfstring{\texttt{LandR} vegetation parameter
estimates for Bird Conservation Region (BCR)
6}{LandR vegetation parameter estimates for Bird Conservation Region (BCR) 6}}\label{landr-vegetation-parameter-estimates-for-bird-conservation-region-bcr-6}

maxB, aNPP and SEP per species and ecodistrict-land cover combination

\texttt{Boreal\_LBMRDataPrep} uses available data on species percent
cover (\% cover), stand age, total biomass, ecodistricts and land cover
class (LCC) (see \texttt{LandR} documentation for product links) to
estimate species-specific parameters of maximum biomass (maxB), seed
establishment probability (SEP) and aboveground net primary productivity
(aNPP), and to build initial unique communities to populate the
landscape and start simulating vegetation dynamics. The selection of
tree species entering the simulation was based on species cover for the
study area from the kNN Species Map Layers available from the Canadian
Forest Service (\url{http://tree.pfc.forestry.ca/}). Species A B
C\ldots{} where selected after removing species that occupied
\textless{} 5 pixels in the study area. Prior to parameter estimation,
incongruencies between stand age, \% cover and total biomass datasets
were addressed and species-specific biomasses were estimated. For
instance, we detected pixels with 0 \% cover across all species layers,
but where stand age and biomass had \textgreater{} 0 values. Given that
biomass and stand age are generally more problematic in terms of
estimation than \% cover, we decided to trust \% cover estimates and
thus assigned 0 total biomass and 0 stand age to any pixel where no
cover was detected. In pixels where the sum of \% cover across species
was \textgreater{}100\%, species \% covers were adjusted so that they
totalled to 100\%. Biomass per species was then calculated per pixel as
a fraction of total biomass according to the species \% cover in that
pixel. Finally, stand age also had to be corrected with respect to
species longevity parameters (obtained from LANDIS-II species traits
tables). This was achieved by fitting a statistical model relating
``correct'' age observations (those already corrected for 0 cover and
that were not \textgreater{} longevity) against the interaction of
observed biomass (totalB) and species identity (speciesCode) and \%
cover, accounting for the random effect of combination of ecodistrict
and LCC (ecoregionCode):

\[age ~ totalB * speciesCode + % cover + (1 | ecoregionCode)\] {[}Eq.
1{]}

This model was then used to predict missing age values, bounded to 0 on
the lower limit.

Parameters maxB and aNPP were then estimated from a linear mixed effects
model reflecting the response of species-specific biomass (B) to the
interaction between age (on the log scale, logAge) and species and \%
cover and species, accounting for the random effect of ecoregionGroup on
the calculated slopes (per species) and intercepts:

\[B ~ logAge * speciesCode + cover * speciesCode + (speciesCode | ecoregionGroup)\]
{[}Eq. 2{]}

Estimates of SEP based on a generalized mixed effects model relating \%
cover and species, accounting for the random effect of ecoregionGroup on
the intercepts. In this case, species \% cover was treated as the number
of times a species was observed (no. of pixels with cover \textgreater{}
0) per ecoregionGroup, thus following a binomial distribution that was
account for in the model with a logit link function:

\[logit(pi) ~ speciesCode + (1 | ecoregionGroup)\] {[}Eq. 3{]}

where \emph{pi} is the probability of finding a species (cover
\textgreater{} 0) in an ecoregionGroup, in other words the proportion of
pixels that it occupies.

For both models, coefficients were estimated by maximum likelihood and
model fit was calculated as the proportion of explained variance
explained by fixed effects only (marginal r\textsuperscript{2}) and by
the entire model (conditional r\textsuperscript{2}). For the biomass
model, marginal and conditional r\textsuperscript{2} were 0.70 and 0.77,
respectively, and for the \% cover model (Eq. 3) they were 0.18 and
0.26.

To estimate maxB we predicted biomass for unique combinations of species
and ecoregion code assuming maximum age (i.e.~longevity). aNPP was then
calculated per species and ecoregionGroup as maxB/30 (following
LANDIS-II). Before estimating SEP, the predicted values of \% cover per
ecoregionGroup per species (obtained from Eq. 3) where temporally
integrated to reflect the simulation timescale (10 years) as 1 - (1 -
predictedCover)\^{}10. These estimates were then multiplied by 1 -
resprouting probability of each species, so that SEP reflects only the
probability of seed germination (note that resprouting probabilities
were taken from the LANDIS-II species trait tables). Remaining
parameters, i.e.~species traits, were obtained from LANDIS-II trait data
(see above for reference) and can be consulted in Table A1

Other parameters {[}Table A1: Species trait values and estimated values
of seed establishment probability (establishprob), maximum biomass
(maxB) and maximum aboveground net primary productivity (maxANPP) by
combination of ecodistrict and land-cover class (ecoregionGroup){]}

The \texttt{Biomass\_regeneration} module simulates post-disturbance
regeneration, in this case after fire events, assuming stand-replacing
fires. In each burnt pixel, the module resets pixel biomass to 0 and
activates post-fire resprouting and, or, serotiny depending on species'
abilities to resprout, their seed establishment probabilities (SEP) in
that pixel (i.e.~the pixel's ecodistrict and land-cover classes), and
their tolerance to shading conditions (which, in this case, is no shade
given that biomass is totally removed after fire) (see Table A1 for
trait values)). The module algorithm first assesses for which species
serotiny will be activated according to shading and SEP (light-loving
species and higher SEP will increase the probability of serotiny being
activated). It then assesses for which resprouter species resprouting
will be activated, depending on whether they are are within resprouting
age limits, shading and resprouting probability (again, light-loving
species and higher resprouting probability will increase the probability
of resprouting being activated). For any given pixel, resprouting is
limited to resprouter species for whom serotiny was not activated. This
provides some advantage to serotinous species which would otherwise be
quickly shaded and out-competed by resprouters.


\end{document}
